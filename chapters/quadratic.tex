\section{Quadratic Equations}

\subsection{Definition}
Quadratic equations are polynomial equations of the form:
\[
    ax^2 + bx + c = 0
\]
where \(a\), \(b\), and \(c\) are constants, and \(a \neq 0\).

\subsection{The Quadratic Formula}
The quadratic formula provides a method to solve any quadratic equation:
\[
    x = \frac{-b \pm \sqrt{b^2 - 4ac}}{2a}
\]
Here, the term \(b^2 - 4ac\) is called the \textbf{discriminant}.

\subsubsection{Steps to Solve Using the Quadratic Formula}
\begin{enumerate}
    \item Identify the coefficients \(a\), \(b\), and \(c\) from the equation.
    \item Compute the discriminant, \(D = b^2 - 4ac\).
    \item Evaluate the roots using the quadratic formula:
          \[
              x_1 = \frac{-b + \sqrt{D}}{2a}, \quad x_2 = \frac{-b - \sqrt{D}}{2a}
          \]
\end{enumerate}

\subsection{Nature of Roots}
The discriminant determines the nature of the roots:
\begin{itemize}
    \item If \(D > 0\), the equation has two distinct real roots.
    \item If \(D = 0\), the equation has one real root (a repeated root).
    \item If \(D < 0\), the equation has two complex roots.
\end{itemize}

\subsection{Example}
Solve the quadratic equation:
\[
    2x^2 - 4x - 6 = 0
\]

\subsubsection{Solution}
\begin{enumerate}
    \item Identify coefficients: \(a = 2\), \(b = -4\), \(c = -6\).
    \item Compute the discriminant:
          \[
              D = (-4)^2 - 4(2)(-6) = 16 + 48 = 64
          \]
    \item Use the quadratic formula:
          \[
              x = \frac{-(-4) \pm \sqrt{64}}{2(2)} = \frac{4 \pm 8}{4}
          \]
    \item Calculate the roots:
          \[
              x_1 = \frac{4 + 8}{4} = 3, \quad x_2 = \frac{4 - 8}{4} = -1
          \]
\end{enumerate}
Thus, the solutions are \(x = 3\) and \(x = -1\).