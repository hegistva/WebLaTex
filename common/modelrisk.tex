\section{Model Risk Management}
\label{sec:modelrisk}

\subsection{Introduction}
Model risk management has become a critical area of focus for banks due to the increasing reliance on models for decision-making and regulatory compliance. Regulatory bodies have established guidelines and requirements to ensure that banks effectively manage the risks associated with their models.

\subsection{Regulatory Framework}
The regulatory framework for model risk management is primarily driven by guidelines such as the Federal Reserve's SR 11-7 in the United States and similar regulations in other jurisdictions. These frameworks emphasize the importance of model validation, governance, and independent review.

\subsubsection{SR 11-7 Overview}
SR 11-7, issued by the Federal Reserve, provides comprehensive guidance on managing model risk. It defines model risk as the potential for adverse consequences arising from decisions based on incorrect or misused models. The guidance outlines two key sources of model risk:
\begin{itemize}
    \item The model may have fundamental errors and produce inaccurate outputs.
    \item The model may be used inappropriately or beyond its intended scope.
\end{itemize}

\subsubsection{Key Principles of SR 11-7}
SR 11-7 establishes several principles for effective model risk management:
\begin{enumerate}
    \item \textbf{Model Development and Implementation:} Models should be developed with sound theoretical foundations and robust methodologies.
    \item \textbf{Model Validation:} Independent validation should be conducted to assess the model's performance, limitations, and assumptions.
    \item \textbf{Governance and Oversight:} Institutions must establish a governance framework to oversee model risk management, including roles and responsibilities.
    \item \textbf{Ongoing Monitoring:} Models should be monitored continuously to ensure they remain accurate and relevant over time.
\end{enumerate}

\subsection{Key Requirements}
\subsubsection{Model Governance}
Banks are required to establish robust governance frameworks to oversee model development, implementation, and usage. This includes defining roles and responsibilities, as well as ensuring accountability.

\subsubsection{Model Validation}
Regulations mandate that banks perform independent validation of their models to assess their accuracy, reliability, and limitations. This process should be conducted periodically and whenever significant changes are made to the models.

\subsubsection{Documentation and Reporting}
Comprehensive documentation of models, including their assumptions, methodologies, and limitations, is a key requirement. Banks must also provide regular reports to senior management and regulators on model performance and associated risks.

\subsection{Conclusion}
Adhering to regulatory requirements for model risk management is essential for banks to mitigate potential risks and maintain compliance. By implementing strong governance, validation, and documentation practices, banks can ensure the effective management of their models.