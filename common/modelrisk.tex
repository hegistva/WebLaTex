\section{Model Risk Management}
\label{sec:modelrisk}

\subsection{Introduction}
Model risk management has become a critical area of focus for banks due to the increasing reliance on models for decision-making and regulatory compliance. Regulatory bodies have established guidelines and requirements to ensure that banks effectively manage the risks associated with their models.

\subsection{Regulatory Framework}
The regulatory framework for model risk management is primarily driven by guidelines such as the Federal Reserve's SR 11-7 in the United States and similar regulations in other jurisdictions. These frameworks emphasize the importance of model validation, governance, and independent review.

\subsubsection{SR 11-7: Supervisory Guidance on Model Risk Management}
SR 11-7, issued by the Federal Reserve, provides comprehensive guidance on managing model risk. It outlines two key aspects of model risk:
\begin{itemize}
    \item \textbf{Model Development and Implementation:} Ensuring that models are conceptually sound, well-documented, and implemented correctly.
    \item \textbf{Model Validation:} Conducting independent validation to assess the accuracy, reliability, and limitations of models.
\end{itemize}
The guidance also emphasizes the need for strong governance, including clear roles and responsibilities, regular reporting to senior management, and ongoing monitoring of model performance.

\subsection{Key Requirements}
\subsubsection{Model Governance}
Banks are required to establish robust governance frameworks to oversee model development, implementation, and usage. This includes defining roles and responsibilities, as well as ensuring accountability.

\subsubsection{Model Validation}
Regulations mandate that banks perform independent validation of their models to assess their accuracy, reliability, and limitations. This process should be conducted periodically and whenever significant changes are made to the models.

\subsubsection{Documentation and Reporting}
Comprehensive documentation of models, including their assumptions, methodologies, and limitations, is a key requirement. Banks must also provide regular reports to senior management and regulators on model performance and associated risks.

\subsection{Conclusion}
Adhering to regulatory requirements for model risk management is essential for banks to mitigate potential risks and maintain compliance. By implementing strong governance, validation, and documentation practices, banks can ensure the effective management of their models.