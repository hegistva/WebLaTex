% This is a simple sample document.  For more complicated documents take a look in the exercise tab. Note that everything that comes after a % symbol is treated as comment and ignored when the code is compiled.

\documentclass{article} % \documentclass{} is the first command in any LaTeX code.  It is used to define what kind of document you are creating such as an article or a book, and begins the document preamble

\usepackage{amsmath} % \usepackage is a command that allows you to add functionality to your LaTeX code

\title{Simple Sample} % Sets article title
\author{My Name} % Sets authors name
\date{\today} % Sets date for date compiled

% The preamble ends with the command \begin{document}
\begin{document} % All begin commands must be paired with an end command somewhere
\maketitle % creates title using information in preamble (title, author, date)

\section{Hello World!} % creates a section

\textbf{Hello World!} Today I am learning \LaTeX. %notice how the command will end at the first non-alphabet charecter such as the . after \LaTeX
\LaTeX{} is a great program for writing math. I can write in line math such as $a^2+b^2=c^2$ %$ tells LaTexX to compile as math
. I can also give equations their own space:
\begin{equation} % Creates an equation environment and is compiled as math
  \gamma^2+\theta^2=\omega^2
\end{equation}
If I do not leave any blank lines \LaTeX{} will continue  this text without making it into a new paragraph.  Notice how there was no indentation in the text after equation (1).
Also notice how even though I hit enter after that sentence and here $\downarrow$
\LaTeX{} formats the sentence without any break.  Also   look  how      it   doesn't     matter          how    many  spaces     I put     between       my    words.

For a new essay I can leave a blank space in my code. Some Change.

\section{Quadratic Equations}

\subsection{Definition}
Quadratic equations are polynomial equations of the form:
\[
    ax^2 + bx + c = 0
\]
where \(a\), \(b\), and \(c\) are constants, and \(a \neq 0\).

\subsection{The Quadratic Formula}
The quadratic formula provides a method to solve any quadratic equation:
\[
    x = \frac{-b \pm \sqrt{b^2 - 4ac}}{2a}
\]
Here, the term \(b^2 - 4ac\) is called the \textbf{discriminant}.

\subsubsection{Steps to Solve Using the Quadratic Formula}
\begin{enumerate}
    \item Identify the coefficients \(a\), \(b\), and \(c\) from the equation.
    \item Compute the discriminant, \(D = b^2 - 4ac\).
    \item Evaluate the roots using the quadratic formula:
          \[
              x_1 = \frac{-b + \sqrt{D}}{2a}, \quad x_2 = \frac{-b - \sqrt{D}}{2a}
          \]
\end{enumerate}

\subsection{Nature of Roots}
The discriminant determines the nature of the roots:
\begin{itemize}
    \item If \(D > 0\), the equation has two distinct real roots.
    \item If \(D = 0\), the equation has one real root (a repeated root).
    \item If \(D < 0\), the equation has two complex roots.
\end{itemize}

\subsection{Example}
Solve the quadratic equation:
\[
    2x^2 - 4x - 6 = 0
\]

\subsubsection{Solution}
\begin{enumerate}
    \item Identify coefficients: \(a = 2\), \(b = -4\), \(c = -6\).
    \item Compute the discriminant:
          \[
              D = (-4)^2 - 4(2)(-6) = 16 + 48 = 64
          \]
    \item Use the quadratic formula:
          \[
              x = \frac{-(-4) \pm \sqrt{64}}{2(2)} = \frac{4 \pm 8}{4}
          \]
    \item Calculate the roots:
          \[
              x_1 = \frac{4 + 8}{4} = 3, \quad x_2 = \frac{4 - 8}{4} = -1
          \]
\end{enumerate}
Thus, the solutions are \(x = 3\) and \(x = -1\). % Includes the content from quadratic.tex

\section{Itô's Lemma}

Itô's Lemma is a fundamental result in stochastic calculus, widely used in financial mathematics and other fields involving stochastic processes. It provides a way to compute the differential of a function of a stochastic process, particularly when the process follows an Itô drift-diffusion model.

For a function \( f(t, X_t) \), where \( X_t \) is a stochastic process satisfying the stochastic differential equation:
\[
    dX_t = \mu(t, X_t) \, dt + \sigma(t, X_t) \, dW_t,
\]
Itô's Lemma states that:
\[
    df(t, X_t) = \frac{\partial f}{\partial t} \, dt + \frac{\partial f}{\partial x} \, dX_t + \frac{1}{2} \frac{\partial^2 f}{\partial x^2} \sigma^2(t, X_t) \, dt.
\]

This result is crucial for modeling and analyzing systems influenced by randomness, such as option pricing in financial markets. % Includes the content from ito.tex

\section{Model Risk Management}
\label{sec:modelrisk}

\subsection{Introduction}
Model risk management has become a critical area of focus for banks due to the increasing reliance on models for decision-making and regulatory compliance. Regulatory bodies have established guidelines and requirements to ensure that banks effectively manage the risks associated with their models.

\subsection{Regulatory Framework}
The regulatory framework for model risk management is primarily driven by guidelines such as the Federal Reserve's SR 11-7 in the United States and similar regulations in other jurisdictions. These frameworks emphasize the importance of model validation, governance, and independent review.

\subsubsection{SR 11-7 Overview}
SR 11-7, issued by the Federal Reserve, provides comprehensive guidance on managing model risk. It defines model risk as the potential for adverse consequences arising from decisions based on incorrect or misused models. The guidance outlines two key sources of model risk:
\begin{itemize}
    \item The model may have fundamental errors and produce inaccurate outputs.
    \item The model may be used inappropriately or beyond its intended scope.
\end{itemize}

\subsubsection{Key Principles of SR 11-7}
SR 11-7 establishes several principles for effective model risk management:
\begin{enumerate}
    \item \textbf{Model Development and Implementation:} Models should be developed with sound theoretical foundations and robust methodologies.
    \item \textbf{Model Validation:} Independent validation should be conducted to assess the model's performance, limitations, and assumptions.
    \item \textbf{Governance and Oversight:} Institutions must establish a governance framework to oversee model risk management, including roles and responsibilities.
    \item \textbf{Ongoing Monitoring:} Models should be monitored continuously to ensure they remain accurate and relevant over time.
\end{enumerate}

\subsection{Key Requirements}
\subsubsection{Model Governance}
Banks are required to establish robust governance frameworks to oversee model development, implementation, and usage. This includes defining roles and responsibilities, as well as ensuring accountability.

\subsubsection{Model Validation}
Regulations mandate that banks perform independent validation of their models to assess their accuracy, reliability, and limitations. This process should be conducted periodically and whenever significant changes are made to the models.

\subsubsection{Documentation and Reporting}
Comprehensive documentation of models, including their assumptions, methodologies, and limitations, is a key requirement. Banks must also provide regular reports to senior management and regulators on model performance and associated risks.

\subsection{Conclusion}
Adhering to regulatory requirements for model risk management is essential for banks to mitigate potential risks and maintain compliance. By implementing strong governance, validation, and documentation practices, banks can ensure the effective management of their models. % Includes the content from modelrisk.tex

\end{document} % This is the end of the document